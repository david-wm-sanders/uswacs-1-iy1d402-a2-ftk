% !TEX TS-program = pdflatex
\documentclass[12pt]{article}

% Package the packages
\usepackage[T1]{fontenc}
\usepackage[utf8]{inputenc}
\usepackage{lmodern}
\usepackage[a4paper, margin=0.75in]{geometry}
\usepackage{enumitem}
\usepackage[colorlinks=true, linkcolor=black, citecolor=black, urlcolor=blue]{hyperref}
\usepackage[nottoc,numbib]{tocbibind}
\usepackage[round]{natbib}
\usepackage{pdfpages}
\usepackage{fancyvrb}
% \usepackage[parfill]{parskip}
% \usepackage{titlesec}
\usepackage{listings}
% \usepackage{graphicx}
% \usepackage{float}
% \usepackage[british]{babel}
% \usepackage{csquotes}
\usepackage{tabularx}
% \usepackage{makecell}
% -

% Configuration
% Change font to Palatino
\renewcommand{\rmdefault}{ppl}
% Change the list item spacing
\setlist{noitemsep}
% Set the bibliography style
\bibliographystyle{usw}
% Set up better chapter titling
% \titleformat{\chapter}{\normalfont\LARGE\bfseries}{\thechapter.}{12pt}{}
% \titlespacing{\chapter}{0pt}{0pt}{12pt}
% \titleformat{\section}{\normalfont\Large\bfseries}{\thesection}{12pt}{}
% \titlespacing{\section}{0pt}{0pt}{6pt}
% \titleformat{name=\section,numberless}{\normalfont\Large\bfseries}{}{0pt}{}
% \titleformat{\subsection}{\normalfont\large\bfseries}{\thesubsection}{12pt}{}
% \titleformat{name=\subsection,numberless}{\normalfont\large\bfseries}{}{0pt}{}
% \titlespacing{\subsection}{0pt}{0pt}{3pt}
% Set up the images path
% \graphicspath{ {deliverables/screenshots/} }
% -

% Custom functions
% Set up an inline todo command
\newcommand{\todo}[1]{\textcolor{red}{todo: #1}}
% Set up a todo environment
\newenvironment{todoenv}
  {\color{red}todo:}
  {\color{black}}
% Set up a terminal command block
\definecolor{light-gray}{gray}{0.95}
\newcommand{\term}[1]{\colorbox{light-gray}{\texttt{#1}}}

% Definitions
% \title{IY1D402\thanks{National Cyber Security Academy, University of South Wales, UK}\\{\textit{\small Cyber Security Tools And Practices}}\\SCADA Investigative Report}
\title{IY1D402{\textit{\small \\Cyber Security Tools And Practices}}\\FTK Forensics Report}
\author{David Sanders\\{\LARGE 17135397}}
\date{\today}
% -

% Document
\begin{document}

% Cover page setup
\maketitle
\pagebreak
% \fontsize{11.5pt}{11.5pt}\selectfont
\tableofcontents
% \fontsize{12pt}{12pt}\selectfont
% -

% Introduction
\pagebreak
\section{ACPO}
The \textit{ACPO Good Practice for Digital Evidence, Version 5} has been prepared by the Association of Chief Police Officers Lead for e-Crime to assist those involved, not just law enforcement, in investigating crime and cyber security incidents.

There are four guiding principles:
\paragraph{Principle 1}
``No action taken by law enforcement agencies, persons employed within those agencies or their agents should change data which may subsequently be relied upon in court.''
\paragraph{Principle 2}
``In circumstances where a person finds it necessary to access original data, that person must be competent to do so and be able to give evidence explaining the relevance and the implications of their actions.''
\paragraph{Principle 3}
``An audit trail or other record of all processes applied to digital evidence should be created and preserved. An independent third party should be able to examine those processes and achieve the same result.''
\paragraph{Principle 4}
``The person in charge of the investigation has overall responsibility for ensuring that the law and these principles are adhered to.''\\

In applying these principles, it is important to be aware of the following. The evidence which is produced in court is to be no more and no less at the time of presentation than it was when it was first taken into the possession of law enforcement and the onus is on the prosecution to show that this is the case. Operating systems and other programs frequently alter, add to, and delete the contents of electronic storage and this may happen automatically without the user being aware that the data has changed. It follows that an image should be made of the original data - creating an image will ensure that the original data is preserved so that an independent third-party can re-examine it if required. Images may be physical/logical block images of the entire device or a logical file image containing partial or selective data. Where data is only stored remotely in a location that is inaccessible, obtaining an image may be impossible. It is essential in such cases that only a person who is competent and appropriately trained accesses and recovers data. Objectivity must be displayed in a court of law as well as the integrity of evidence. Audit trails must be kept to demonstrate how evidence has been recovered -- all processes must be shown.
\pagebreak
\section{Evidence Listing and Reporting}
\subsection{Example Artefact Template}
\subsubsection{1}
\begin{table}[h!]
\centering
% \caption{My caption}
% \label{my-label}
\ttfamily
% \small
\newcolumntype{k}{p{3cm}}
\newcolumntype{v}{X}
\newcolumntype{t}{p{\textwidth - 4\tabcolsep - 3cm - 3\arrayrulewidth}}
\begin{tabularx}{1\textwidth}{|k|v|k|v|}
\hline
FTK No.               & x                       & Evidence No.  & x  \tabularnewline \hline
Filename              & \multicolumn{3}{t|}{x}                       \tabularnewline \hline
Hash (MD5)            & \multicolumn{3}{t|}{x}                       \tabularnewline \hline
Hash (SHA1)           & \multicolumn{3}{t|}{x}                       \tabularnewline \hline
Date Created          & \multicolumn{3}{t|}{x}                       \tabularnewline \hline
Date Modified         & \multicolumn{3}{t|}{x}                       \tabularnewline \hline
Date Accessed         & \multicolumn{3}{t|}{x}                       \tabularnewline \hline
Is deleted?           & \multicolumn{3}{t|}{x}                       \tabularnewline \hline
Size (L/P)            & \multicolumn{3}{t|}{x}                       \tabularnewline \hline
File path             & \multicolumn{3}{t|}{x}                       \tabularnewline \hline
Evidence Description  & \multicolumn{3}{t|}{x}                       \tabularnewline \hline
Technique used        & \multicolumn{3}{t|}{x}                       \tabularnewline \hline
Reason for selection  & \multicolumn{3}{t|}{x}                       \tabularnewline \hline
\end{tabularx}
\end{table}

\begin{table}[h!]
\centering
% \caption{My caption}
% \label{my-label}
\ttfamily
% \small
\newcolumntype{k}{p{3cm}}
\newcolumntype{v}{X}
\newcolumntype{t}{p{\textwidth - 4\tabcolsep - 3cm - 3\arrayrulewidth}}
\begin{tabularx}{1\textwidth}{|k|v|k|v|}
\hline
FTK No.                & 17135397\_IY1D402\_CW2  &  Evidence No.  & DS01  \tabularnewline \hline
Filename               & \multicolumn{3}{t|}{helloworld.txt}  \tabularnewline \hline
Hash (MD5)             & \multicolumn{3}{t|}{4262ff3e3c24530bd4c5c97c0e2df4db}  \tabularnewline \hline
Hash (SHA1)            & \multicolumn{3}{t|}{d995591deb52c94ac928a8f720fbca55af01d945}  \tabularnewline \hline
Date Created           & \multicolumn{3}{t|}{28/04/2018 02:47}  \tabularnewline \hline
Date Modified          & \multicolumn{3}{t|}{28/04/2018 02:47}  \tabularnewline \hline
Date Accessed          & \multicolumn{3}{t|}{28/04/2018 02:47}  \tabularnewline \hline
Is deleted?            & \multicolumn{3}{t|}{No}  \tabularnewline \hline
Size (L/P)             & \multicolumn{3}{t|}{10 bytes / 10 bytes}  \tabularnewline \hline
File path              & \multicolumn{3}{t|}{C:\textbackslash Users\textbackslash david\textbackslash Documents\textbackslash some\textbackslash sub\textbackslash folder}  \tabularnewline \hline
Evidence Description   & \multicolumn{3}{t|}{This file was discovered in the Documents folder inside the user's home directory. It contains vital evidence which links the suspect definitively to malfeasant activities. This text is just filler designed to test \LaTeX\ rendering of large text bodies inside tabularx environments.}  \tabularnewline \hline
Technique used         & \multicolumn{3}{t|}{This file was discovered in the Documents folder inside the user's home directory. It contains vital evidence which links the suspect definitively to malfeasant activities. This text is just filler designed to test \LaTeX\ rendering of large text bodies inside tabularx environments.}  \tabularnewline \hline
Reason for selection   & \multicolumn{3}{t|}{This file was discovered in the Documents folder inside the user's home directory. It contains vital evidence which links the suspect definitively to malfeasant activities. This text is just filler designed to test \LaTeX\ rendering of large text bodies inside tabularx environments.}  \tabularnewline \hline
\end{tabularx}
\end{table}

\subsection{Example Hash Block Template}
\begin{table}[h!]
\ttfamily\footnotesize
\setlength{\tabcolsep}{0.1cm}
\newcolumntype{k}{p{1cm}}
\newcolumntype{v}{c}
\begin{tabular}{|k|v|v|v|v|v|v|v|v|v|v|v|v|v|v|v|v|v|v|v|v|v|v|v|v|v|v|v|v|v|v|v|v|v|v|v|v|v|v|v|v|}
\hline
MD5 & 4 & 2 & 6 & 2 & f & f & 3 & e & 3 & c & 2 & 4 & 5 & 3 & 0 & b & d & 4 & c & 5 & c & 9 & 7 & c & 0 & e & 2 & d & f & 4 & d & b & - & - & - & - & - & - & - & - \tabularnewline \hline
SHA1 & d & 9 & 9 & 5 & 5 & 9 & 1 & d & e & b & 5 & 2 & c & 9 & 4 & a & c & 9 & 2 & 8 & a & 8 & f & 7 & 2 & 0 & f & b & c & a & 5 & 5 & a & f & 0 & 1 & d & 9 & 4 & 5 \tabularnewline \hline
\end{tabular}
\end{table}
{
\ttfamily\tiny
\vspace{-4em}
\begin{flushright}
MD5:4262ff3e3c24530bd4c5c97c0e2df4db~SHA1:d995591deb52c94ac928a8f720fbca55af01d945
\end{flushright}
}
\pagebreak
\section{Passwords Recovered}


% ------------------------------------------------------------------------------
% \begin{table}[h!]
%   \centering
%   \begin{tabular}{|r l|}
%     \hline
%     IP address: & 192.168.254.132 \\
%     \hline
%     Operating system: & Linux Ubuntu 4.13.0-17-generic x86\_64 \\
%     \hline
%     Open ports: & 80 \textit{[http]}, 442 \textit{[ssh; OpenSSH 7.5p1 Ubuntu 10]} \\
%     \hline
%     Closed ports: & 443 \textit{[https]} \\
%     \hline
%     Price's groups: & phillip adm cdrom sudo dip www-data plugdev lpadmin sambashare \\
%     \hline
%   \end{tabular}
%   \caption{Information on Price's machine and account}
%   \label{table:pricemachineinfo}
% \end{table}

% \begin{Verbatim}[frame=leftline]
% Loaded 393 passwords from 'passwords.txt'
% (001/393; 00.25%) Password != '0'
% (002/393; 00.51%) Password != 'scorpio'
% (003/393; 00.76%) Password != 'buddy'
% ...
% (222/393; 56.49%) Password != 'trustno1'
% (223/393; 56.74%) Password != 'newyork'
% (224/393; 57.00%) Password == 'qwertyuiop'
% Took 0.96 minutes to check 224/393 passwords at a rate of 3.88pw/s.
% Performing recon...
% Logging out.
% \end{Verbatim}

% \begin{Verbatim}[frame=leftline]
% [User]
% Language=en_GB
% FormatsLocale=en_GB.UTF-8
% XSession=
% SystemAccount=true
%
% [InputSource0]
% xkb=gb
% \end{Verbatim}
% ------------------------------------------------------------------------------

% BIBLIOGRAPHY/REFERENCES
\pagebreak
% nocited refs
% \nocite{example:referenceid:here}

% Insert references section, left aligned
\begin{flushleft}
  \bibliography{references}
\end{flushleft}


% APPENDICES
% Appendix~\ref{app:screenshots:7}
% \appendix

% \pagebreak
% \chapter{Screenshot deliverables}
% \section{\texttt{ifconfig}, \texttt{arp-scan}, and \texttt{nmap}}
% \label{app:screenshots:1}
% \begin{figure}[H]
%   \centering
%   \includegraphics[width=0.7\paperheight, angle=-90]{It's_Elementary_my_dear_Watson!-2017-12-06-13-41-10}
% \end{figure}
% \pagebreak

% \pagebreak
% \chapter{File deliverables}
% \section{\texttt{recon.txt}}
% \label{app:files:recon}
% \lstinputlisting[frame=single, basicstyle=\small\ttfamily, showstringspaces=false, breaklines=true, postbreak=\mbox{\textcolor{gray}{$\hookrightarrow$}\space}]{deliverables/files/recon.txt}


\end{document}
